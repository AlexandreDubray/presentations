\documentclass[10pt]{beamer}

\usepackage{custom}

\title{Mining Constrained Regions of Interest: An optimization approach}
\date{}
\author{Alexandre Dubray \and Guillaume Derval \and Siegfried Nijssen \and Pierre Schaus}
\institute{}
% \titlegraphic{\hfill\includegraphics[height=1.5cm]{logo.pdf}}

\begin{document}

\maketitle

\begin{frame}{Table of contents}
  \setbeamertemplate{section in toc}[sections numbered]
  \tableofcontents%[hideallsubsections]
\end{frame}

\section{Introduction}

\begin{frame}{Motivations}
\begin{itemize}
    \item The amount of spatiotemporal data is exploding (smartphone applications, sport devices, fleet management, etc)
    \item There is a need to process more efficiently these data
    \item Rewrite the raw trajectories (GPS points) as sequence of \emph{Regions of Interest} (ROI)
    \item Multiple applications:
        \begin{itemize}
            \item Trajectory pattern mining
            \item Next location prediction
            \item Urban management
            \item ...
        \end{itemize}
\end{itemize}
\end{frame}

\begin{frame}{Preparation of the data}
    \begin{enumerate}
        \item Divide the map with a $N \times M$ grid of square cells.
        \item Assign a density value to each cell. A cell is dense if its density is above a threshold
        \item Express the ROI as an aggregation of dense cells
    \end{enumerate}

    Multiple possibility for the density
    \begin{itemize}
        \item Number of trajectories that cross the cell (with and without interpolation)
        \item Number of trajectories that stayed at least 10 minutes in a cell
        \item etc.
    \end{itemize}
\end{frame}

\begin{frame}{Example of ROIs}
\begin{figure}
    \begin{subfigure}{0.49\textwidth}
        \includegraphics[scale=0.2]{figures/map/grid-init.pdf}
        \caption{Initial set of dense cells}
    \end{subfigure}
    \begin{subfigure}{0.49\textwidth}
        \includegraphics[scale=0.2]{figures/map/grid-ilp.pdf}
        \caption{Solution found by our method}
    \end{subfigure}
\end{figure}
\end{frame}

\section{PopularRegion}

\begin{frame}{Execution of the algorithm}
    \begin{figure}
        \centering
        \includegraphics{figures/running-example/running-ex-init.pdf}
    \end{figure}
\end{frame}

\newcounter{loop}
\forloop{loop}{1}{\value{loop}<7}{\figurepp{\arabic{loop}}}

\begin{frame}{Result of the algorithm}
\begin{figure}
    \begin{subfigure}{0.49\textwidth}
        \includegraphics[scale=0.2]{figures/map/grid-init.pdf}
        \caption{Initial set of dense cells}
    \end{subfigure}
    \begin{subfigure}{0.49\textwidth}
        \includegraphics[scale=0.2]{figures/map/grid-pp.pdf}
        \caption{Solution with $5\%$ min average density}
    \end{subfigure}
\end{figure}
\end{frame}

\begin{frame}{Advantages and disadvantages}
    \begin{itemize}
        \item Scalable
        \item Intuitive and good results for most configurations
    \end{itemize}
    But...
    \begin{itemize}
        \item No formalization of the output
        \item Only rectangular regions
        \item Does not easily accept background knowledge
    \end{itemize}
\end{frame}

\section{Our method}

\begin{frame}{Outline}
    \begin{enumerate}
        \item Generate a set of candidate ROI
        \begin{itemize}
            \item Can have any shape
            \item Impose \emph{intra-ROI} constraints
        \end{itemize}
        \item Select from the candidates $K$ final ROIs
        \begin{itemize}
            \item Found by an optimization problem
            \item Impose \emph{inter-ROI} constraints
        \end{itemize}
    \end{enumerate}
\end{frame}

\begin{frame}{ROIs as an encoder}

\begin{columns}[T, onlytextwidth]
    \column{.5\textwidth}
    \begin{itemize}
        \item The ROIs encode the dense status of the cells
        \item Example of encoding with two rectangles
        \begin{itemize}
            \item 1 dense cells is not covered
            \item 4 non-dense cells are covered
            \item The encoding makes 5 errors
        \end{itemize}
        \item We prefer encoding with less errors
    \end{itemize}
    
    \column{.5\textwidth}
    \begin{figure}
        \centering
        \includegraphics[scale=0.5]{figures/running-example/ILP/running-ex-ilp1.pdf}
    \end{figure}
\end{columns}

\end{frame}

\begin{frame}{Formalization of the problem (1)}
Some notations:
    \begin{itemize}
        \item Let $\mathcal{G}$ be the grid, $\mathcal{G}^*$ the set of dense cells, $\mathcal{S}$ a set of candidates and $\theta$ the minimum density threshold
        \item $d_i$ (resp. $u_i$) is the number of dense (resp. non-dense) cells covered by the candidate $R_i \in \mathcal{S}$
        \item $K$ is the number of ROIs we want to find
    \end{itemize}
\end{frame}

\begin{frame}{A first optimization model}
\begin{IEEEeqnarray*}{r;l;l'l} %note: one extra col to add space at the right
    \IEEEeqnarraymulticol{4}{l}{\mbox{minimize}\ |\mathcal{G}^*| + \sum_{R_i \in \mathcal{S}} x_i \cdot (u_i-d_i) \label{eq:opti-base}} \label{eq:ext-base2}\\
    \IEEEeqnarraymulticol{4}{l}{\mbox{subject to}}\nonumber\\
    \textstyle\sum_{R_i \in \mathcal{S}} x_i &\leq & K  \label{eq:constr-K}\\
    \textstyle\sum_{R_i \in \mathcal{S} \mid c \in R_i} x_i & \leq & 1 & \forall c \in \mathcal{G} \label{eq:constr-overlap} \\
    x_i & \in & \{0,1\} & \forall R_i \in \mathcal{S} \label{eq:constr-xi}
\end{IEEEeqnarray*}

\end{frame}

\begin{frame}{Formalization of the problem (2)}
    In practice how to set the $K$? Use the Minimum Description Length Principle!
    \begin{itemize}
        \item Let $Sol \subseteq \mathcal{S}$ be a valid solution
        \item Length of the errors:
            $$L(\mathcal{G}\mid Sol) = 2|\mathcal{G}^*| + \sum_{R_i \in Sol} 2(u_i - d_i)$$
        \item Length of the model: $$L(Sol) = \sum_{R_i \in Sol} size(R_i)$$
        \item Minimum Description Length principle tells that the best solution is:
            \begin{equation*}
                \argmin_{Sol \in \mathcal{S}} L(\mathcal{G} \mid Sol) + L(Sol)
            \end{equation*}
    \end{itemize}
\end{frame}

\begin{frame}{The final optimization model}

    \begin{IEEEeqnarray*}{r;l;l'l} %note: one extra col to add space at the right
        \IEEEeqnarraymulticol{4}{l}{\mbox{minimize}\ \sum_{R_i \in \mathcal{S}} x_i \cdot \left(2(u_i-d_i)+size(R_i)\right)} \label{eq:extended-opti}\\
        \IEEEeqnarraymulticol{4}{l}{\mbox{subject to}}\nonumber\\
        \textstyle\sum_{R_i \in \mathcal{S} \mid c \in R_i} x_i & \leq & 1 & \forall c \in \mathcal{G} \label{eq:extended-ctr} \\
        x_i & \in & \{0,1\} & \forall R_i \in \mathcal{S} \label{eq:extended-integer}
    \end{IEEEeqnarray*}
\end{frame}

\begin{frame}{Example}

    \begin{columns}[T, onlytextwidth]
        \column{.5\textwidth}
            \begin{itemize}
                \item $L(\mathcal{S}) = 4 + 4 = 8$
                \item $L(\mathcal{G} \mid \mathcal{S}) = 2\cdot (1 + 4) = 10$
                \item Total length of this model is $8 + 10 = 18$
            \end{itemize}

        \begin{figure}
            \centering
            \includegraphics[scale=0.5]{figures/running-example/MDL/example-1.pdf}
        \end{figure}

        \column{.5\textwidth}

            \begin{itemize}
                \item $L(\mathcal{S}) = 4 + 4 = 8$
                \item $L(\mathcal{G} \mid \mathcal{S}) = 2\cdot (2 + 2) = 8$
                \item Total length of this model is $8 + 8 = 16$
            \end{itemize}

        \begin{figure}
            \centering
            \includegraphics[scale=0.5]{figures/running-example/MDL/example-2.pdf}
        \end{figure}
    \end{columns}
\end{frame}

\begin{frame}{Example with circles}

    \begin{columns}[T, onlytextwidth]
        \column{.5\textwidth}
            \begin{itemize}
                \item $L(\mathcal{S}) = 4 + 4 = 8$
                \item $L(\mathcal{G} \mid \mathcal{S}) = 2\cdot (1 + 4) = 10$
                \item Total length of this model is $8 + 10 = 18$
            \end{itemize}

        \begin{figure}
            \centering
            \includegraphics[scale=0.5]{figures/running-example/MDL/example-1.pdf}
        \end{figure}

        \column{.5\textwidth}

            \begin{itemize}
                \item $L(\mathcal{S}) = 4 + 3 = 7$
                \item $L(\mathcal{G} \mid \mathcal{S}) = 2\cdot (2 + 3) = 10$
                \item Total length of this model is $7 + 10 = 17$
            \end{itemize}

        \begin{figure}
            \centering
            \includegraphics[scale=0.5]{figures/running-example/MDL/example-3.pdf}
        \end{figure}
    \end{columns}
\end{frame}

\begin{frame}{The full method}
    \begin{enumerate}
        \item Generate the set of candidates $\mathcal{S}$ (e.g. enumerate all distinct
            rectangle on the grid)
            \begin{itemize}
                \item Candidate can have any shape
                \item Compute their contribution to the description length
                \item Apply \emph{intra-ROI} constraints to filter the candidate set
            \end{itemize}
        \item Solve the optimization model
            \begin{itemize}
                \item Apply \emph{inter-ROI} constraints
                \item Model these constraint with linear constraints
            \end{itemize}
        \item The result of the optimization (choice of candidates) is the set of ROIs
            we return
    \end{enumerate}
\end{frame}


\section{Experiments}

\begin{frame}{Setup}
    \begin{itemize}
        \item Two versions of our method
            \begin{itemize}
                \item With only rectangular regions
                \item With rectangular and circular regions
            \end{itemize}
        \item Showing results on Kaggle taxis dataset ($\approx$1.6 million trajectories)
        \item Comparing with PopularRegion and OPTICS (when clustering the dense cells)
    \end{itemize}
\end{frame}

\begin{frame}{Execution time}
\begin{table}[!htb]
\centering
{\resizebox{\linewidth}{!}{
\setlength{\tabcolsep}{7pt}
\begin{tabular}{lllllll}
\toprule
Minimum density threshold & \multicolumn{3}{c}{2\%} & \multicolumn{3}{c}{5\%} \\
\cmidrule(lr){2-4} \cmidrule(l){5-7}
Grid side size            & 100 & 150 & 200           & 100 & 150 & 200 \\
\midrule
Number of dense cells ($|\mathcal{G}^*|$) & 571 & 597 & 537 & 230 & 178 & 137 \\
\addlinespace 
Number of ILP candidates                  & 23 814 & 7 779 & 3 399 & 2 880 & 1 232 & 434 \\
ILP optimization time (s)                 & 4.328 & 0.464 & 0.109 & 0.113 & 0.044 & 0.029 \\ 
\addlinespace 
\emph{PopularRegion} run time (s)       & 0.003 & 0.005 & 0.006 & 0.002 & 0.003 & 0.004 \\
\addlinespace 
OPTICS run time (s)                          & 0.209 & 0.222 & 0.200 & 0.084 & 0.065 & 0.051 \\
\bottomrule
\end{tabular}%
}}
\end{table}
\end{frame}

\begin{frame}{Description Length}
    \begin{itemize}
        \item For high density threshold, number of errors becomes similar
        \item ILP-based methods produce smaller models
        \item Overall the Description Length is inferior for ILP-based methods
    \end{itemize}
    \begin{columns}[T, onlytextwidth]

        \column{.5\textwidth}
        \begin{figure}
            \centering
            \includegraphics[scale=0.3]{figures/results/error-rate.pdf}
            \caption{Encoding of the errors}
        \end{figure}

        \column{.5\textwidth}
        \begin{figure}
            \centering
            \includegraphics[scale=0.3]{figures/results/model-length.pdf}
            \caption{Encoding fo the models}
        \end{figure}
    \end{columns}
\end{frame}

\begin{frame}{Robustness to noise}
    \begin{itemize}
        \item Start from a $100 \times 100$ grid
        \item Move every element of the trajectories to a neighboring cell with a probability $p$
        \item Choose the new cell randomly in a square of size $10$ around the initial cell
        \item Recompute solution and compare to initial solution (with min density threshold $5\%$)
    \end{itemize}
    \begin{columns}[T, onlytextwidth]

        \column{.5\textwidth}
        \begin{figure}
            \centering
            \includegraphics[scale=0.3]{figures/results/recall-precision.pdf}
            \caption{Recall and precision}
        \end{figure}

        \column{.5\textwidth}
        \begin{figure}
            \centering
            \includegraphics[scale=0.3]{figures/results/f1.pdf}
            \caption{F1-measure}
        \end{figure}
    \end{columns}
\end{frame}

\begin{frame}{Conclusion}
    \begin{itemize}
        \item We propose an optimization model to find $K$ ROIs from trajectory data
        \item Using the MDL principle we can automatically set the $K$
        \item Our method is more flexible than specific method since it accepts a wide range of constraints
        \item the solutions find by our method are more robust and better generalize the dense cells distribution
        \item The runtime of the ILP becomes reasonnable as long as there is not too much candidates
    \end{itemize}
\end{frame}

\end{document}
